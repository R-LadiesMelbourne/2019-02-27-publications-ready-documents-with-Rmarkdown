\documentclass[12pt,]{article}
\usepackage{lmodern}
\usepackage{amssymb,amsmath}
\usepackage{ifxetex,ifluatex}
\usepackage{fixltx2e} % provides \textsubscript
\ifnum 0\ifxetex 1\fi\ifluatex 1\fi=0 % if pdftex
  \usepackage[T1]{fontenc}
  \usepackage[utf8]{inputenc}
\else % if luatex or xelatex
  \ifxetex
    \usepackage{mathspec}
  \else
    \usepackage{fontspec}
  \fi
  \defaultfontfeatures{Ligatures=TeX,Scale=MatchLowercase}
\fi
% use upquote if available, for straight quotes in verbatim environments
\IfFileExists{upquote.sty}{\usepackage{upquote}}{}
% use microtype if available
\IfFileExists{microtype.sty}{%
\usepackage{microtype}
\UseMicrotypeSet[protrusion]{basicmath} % disable protrusion for tt fonts
}{}
\usepackage[margin=1in]{geometry}
\usepackage{hyperref}
\hypersetup{unicode=true,
            pdftitle={title},
            pdfborder={0 0 0},
            breaklinks=true}
\urlstyle{same}  % don't use monospace font for urls
\usepackage{longtable,booktabs}
\usepackage{graphicx,grffile}
\makeatletter
\def\maxwidth{\ifdim\Gin@nat@width>\linewidth\linewidth\else\Gin@nat@width\fi}
\def\maxheight{\ifdim\Gin@nat@height>\textheight\textheight\else\Gin@nat@height\fi}
\makeatother
% Scale images if necessary, so that they will not overflow the page
% margins by default, and it is still possible to overwrite the defaults
% using explicit options in \includegraphics[width, height, ...]{}
\setkeys{Gin}{width=\maxwidth,height=\maxheight,keepaspectratio}
\IfFileExists{parskip.sty}{%
\usepackage{parskip}
}{% else
\setlength{\parindent}{0pt}
\setlength{\parskip}{6pt plus 2pt minus 1pt}
}
\setlength{\emergencystretch}{3em}  % prevent overfull lines
\providecommand{\tightlist}{%
  \setlength{\itemsep}{0pt}\setlength{\parskip}{0pt}}
\setcounter{secnumdepth}{5}
% Redefines (sub)paragraphs to behave more like sections
\ifx\paragraph\undefined\else
\let\oldparagraph\paragraph
\renewcommand{\paragraph}[1]{\oldparagraph{#1}\mbox{}}
\fi
\ifx\subparagraph\undefined\else
\let\oldsubparagraph\subparagraph
\renewcommand{\subparagraph}[1]{\oldsubparagraph{#1}\mbox{}}
\fi

%%% Use protect on footnotes to avoid problems with footnotes in titles
\let\rmarkdownfootnote\footnote%
\def\footnote{\protect\rmarkdownfootnote}

%%% Change title format to be more compact
\usepackage{titling}

% Create subtitle command for use in maketitle
\newcommand{\subtitle}[1]{
  \posttitle{
    \begin{center}\large#1\end{center}
    }
}

\setlength{\droptitle}{-2em}

  \title{title}
    \pretitle{\vspace{\droptitle}\centering\huge}
  \posttitle{\par}
    \author{Author 1\footnote{Corresponding author:
  \href{mailto:email@email.com}{\nolinkurl{email@email.com}}} \(^1\),
Author 2 \(^1\), Author 3 \(^2\)\\
\(^1\)Affiliation1, \(^2\)Affiliation2}
    \preauthor{\centering\large\emph}
  \postauthor{\par}
    \date{}
    \predate{}\postdate{}
  
\usepackage{float} \floatplacement{figure}{H}
\newcommand{\beginsupplement}{\setcounter{table}{0}  \renewcommand{\thetable}{S\arabic{table}} \setcounter{figure}{0} \renewcommand{\thefigure}{S\arabic{figure}}}
\usepackage{setspace}\doublespacing \usepackage{lineno} \linenumbers

\begin{document}
\maketitle
\begin{abstract}
Your abstract goes here\ldots{}
\end{abstract}

\newpage

\section{Introduction}\label{introduction}

\label{sec:intro}

This is my intro. All of this uses the amazing (Xie 2016) package.

Bookdown extends the syntax provided by R Markdown, allowing automatic
numbering of figures / tables / equations, and cross-referencing them.

\url{https://stackoverflow.com/questions/25824795/how-to-combine-two-rmarkdown-rmd-files-into-a-single-output}

\url{http://landscapeportal.org/blog/2017/09/06/r-markdown-template-for-a-scientific-manuscript/}

\section{Results}\label{results}

\label{sec:res}

Amazing results after introducing the story in Section \ref{sec:intro}.

\subsection{About the Normal
Distribution}\label{about-the-normal-distribution}

\label{sec:res-trees}

\begin{itemize}
\tightlist
\item
  Problem of ``floating environments'' in LaTeX
\end{itemize}

\begin{figure}
\centering
\includegraphics{imgbell-curve-1.pdf}
\caption{\label{fig:bell-curve}Red trees in autumn}
\end{figure}

It's a pretty cool bell curve in Figure \ref{fig:bell-curve}
\ref{fig:bell-curve}. Whatch out for \texttt{\_} in cross-referencing
labels!

If you want to cross-reference figures or tables generated from a code
chunk, please make sure the chunk label only contains alphanumeric
characters (a-z, A-Z, 0-9), slashes (/), or dashes (-).

It is always good to have some Greek letters in your paper! Like
Equation \eqref{eq:binom} below:

\begin{equation} 
  f\left(k\right) = \binom{n}{k} p^k\left(1-p\right)^{n-k}
  \label{eq:binom}
\end{equation}

\subsection{About the scatter plot}\label{about-the-scatter-plot}

\begin{figure}
\centering
\includegraphics{imgscatter-1.pdf}
\caption{\label{fig:scatter}Loud Kookabara}
\end{figure}

\begin{table}[t]

\caption{\label{tab:tree-diamonds}A table of the first 10 rows and three columns of the diamonds data.}
\centering
\begin{tabular}{rlllrrrrrr}
\toprule
carat & cut & color & clarity & depth & table & price & x & y & z\\
\midrule
0.23 & Ideal & E & SI2 & 61.5 & 55 & 326 & 3.95 & 3.98 & 2.43\\
0.21 & Premium & E & SI1 & 59.8 & 61 & 326 & 3.89 & 3.84 & 2.31\\
0.23 & Good & E & VS1 & 56.9 & 65 & 327 & 4.05 & 4.07 & 2.31\\
0.29 & Premium & I & VS2 & 62.4 & 58 & 334 & 4.20 & 4.23 & 2.63\\
0.31 & Good & J & SI2 & 63.3 & 58 & 335 & 4.34 & 4.35 & 2.75\\
\addlinespace
0.24 & Very Good & J & VVS2 & 62.8 & 57 & 336 & 3.94 & 3.96 & 2.48\\
\bottomrule
\end{tabular}
\end{table}

We created the simple Table \ref{tab:tree-diamonds}. See more on tables
and long tables \url{https://bookdown.org/yihui/bookdown/tables.html}.

You can also see a scatter plot in Figure \ref{fig:sup1}.

\subsection{Extra tips: Setup your
bibliography}\label{extra-tips-setup-your-bibliography}

\section{Conclusions}\label{conclusions}

Bla Bla\ldots{}.

\newpage

\hypertarget{refs}{}
\hypertarget{ref-bookdown}{}
Xie, Yihui. 2016. \emph{Bookdown: Authoring Books and Technical
Documents with R Markdown}. Boca Raton, Florida: Chapman; Hall/CRC.
\url{https://github.com/rstudio/bookdown}.


\end{document}
